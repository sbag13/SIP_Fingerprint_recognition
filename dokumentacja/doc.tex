\documentclass[12pt, notitlepage]{article}

\usepackage{geometry}
\usepackage{polski}
\usepackage[utf8]{inputenc}
\usepackage[T1]{fontenc}
\usepackage{enumitem}
\usepackage{graphicx}
\usepackage{float}
\usepackage{listings}
\usepackage{url}
\lstset{basicstyle=\footnotesize\ttfamily,breaklines=true}

\usepackage{etoolbox}
\makeatletter
\patchcmd{\chapter}{\if@openright\cleardoublepage\else\clearpage\fi}{}{}{}
\makeatother

\usepackage[toc]{appendix}
\renewcommand{\appendixtocname}{Dodatki}
\renewcommand\refname{Odwołania}
\usepackage[parfill]{parskip}
% \setlength{\parindent}{0pt}
% \setlength{\parskip}{\baselineskip}

% \usepackage[
%     backend=biber,
%     style=alphabetic,
%     sorting=ynt
%     ]{biblatex}

% \addbibresource{doc.bib}
\geometry{legalpaper, margin=0.8in}

\begin{document}

\begin{titlepage}
    \thispagestyle{empty}
    \title{\textbf{
        \Huge Systemy Inteligentnego Przetwarzania \\
        [1cm]
        \LARGE Rozpoznawanie osób na podstawie odcisków palców przy użyciu sieci Kohonena. 
    }}
    \author{
        Szymon Bagiński \\ 
        Dawid Aksamski \\
        [1cm]
        {\small Prowadzący: Dr inż. Jacek Mazurkiewicz}
    }
    % \author{Szymon Bagiński\thanks{funded by the ShareLaTeX team}}
    \date{Styczeń 2018}
    \maketitle
    \vfill
    % \renewcommand{\chapter}[2]{}
    % \begin{center}
    %     \Large \bfseries\contentsname
    % \end{center}
    % \tableofcontents
    \vfill
\end{titlepage}    

% \chapter{Wstęp}
% \addcontentsline{toc}{chapter}{Wstęp}

\tableofcontents

\newpage
%%%%%%%%%%%%%%%%%%%%%%%%%%%%%%%%%%%%%%%%%%%%%%%%%%%%%%%%%%%%%%%%%%%%%%%%%%%%%%
\section{Wstęp}
%%%%%%%%%%%%%%%%%%%%%%%%%%%%%%%%%%%%%%%%%%%%%%%%%%%%%%%%%%%%%%%%%%%%%%%%%%%%%%

\subsection{Cel projektu}

Celem projektu było stworzenie klasyfikatora odcisków palca, opartego o sieć Kohonena \cite{Kohonen}. Zadaniem sieci było rozpoznanie osoby, której odciski zostały użyte w procesie uczenia, na podstawie obrazu odcisku, który nie był w tym procesie wykorzystany. W temacie zadania nie zostały określone szczegółowe parametry sieci, takie jak na przykład jej pojemność, czy konkretne mechanizmy jakich należałoby użyć do stworzenia klasyfikatora, więc dostosowano je głównie do posiadanych możliwości.

\subsection{Realizacja}

Sieć Kohonena posiada ograniczoną pojemność, która jest równa ilości neuronów. Jej rozmiar jest więc bezpośrednio związany z ilością osób, które mają być rozpoznawane. Podczas pracy nad projektem skorzystano z darmowej bazy odcisków palca, dostępnej pod linkiem:\newline
{
    \parskip=0pt
    % \begin{center}
        \url{
            https://www.neurotechnology.com/download/CrossMatch_Sample_DB.zip
        }
    % \end{center}
}
\newline
Znajduje się tam 408 obrazów odcisków. Wśród nich jest po 8 różnych obrazów tego samego palca, tej samej osoby. Do trenowania klasyfikatora skorzystano z 7 z nich, natomiast jeden pozostawiono do oceny jego skuteczności. Pojemność sieci musiała więc wynosić conajmniej \( \frac{7}{8} \) ilości odcisków, co jest równe 357. W praktyce wykorzystywano siatki neuronów o rozmiarach np. 35x35, co daje 1225 neuronów i powinno w zupełności wystarczyć.

Aby uzyskać dane, które można wprowadzić na wejście sieci Kohonena należy najpierw wydobyć cechy właściwe dla konkretnego odcisku oraz je odpowiednio przygotować. Proces ten został bardziej szczegółowo opisany w pubkcie \ref{sec:extraction}.

Niezbędne jest także wstępne przetworzenie danych wejściowych, tak aby pozbyć się niepotrzebnych informacji, wyeliminować niedoskonałości, jeśli jest to możliwe lub uwydatnić cechy przydatne z punktu widzenia klasyfikacji. Ten krok został opisany w punkcie \ref{sec:preprocesing}.

%%%%%%%%%%%%%%%%%%%%%%%%%%%%%%%%%%%%%%%%%%%%%%%%%%%%%%%%%%%%%%%%%%%%%%%%%%%%%%
\section{Opis implementacji}
%%%%%%%%%%%%%%%%%%%%%%%%%%%%%%%%%%%%%%%%%%%%%%%%%%%%%%%%%%%%%%%%%%%%%%%%%%%%%%

\subsection{Przetwarzanie wstępne obrazu}\label{sec:preprocesing}

\subsection{Wydobywanie cech}\label{sec:extraction}

%%%%%%%%%%%%%%%%%%%%%%%%%%%%%%%%%%%%%%%%%%%%%%%%%%%%%%%%%%%%%%%%%%%%%%%%%%%%%%
\section{Podsumowanie}
%%%%%%%%%%%%%%%%%%%%%%%%%%%%%%%%%%%%%%%%%%%%%%%%%%%%%%%%%%%%%%%%%%%%%%%%%%%%%%

\newpage

\begin{thebibliography}{9}

\bibitem{Kohonen}
\textit{Sieci neuronowe - Klasyfikator Kohonena} [online], Data dostępu: 18.01.2019. 
\newline\url{http://galaxy.agh.edu.pl/~vlsi/AI/koho_t/}


\end{thebibliography} 


\newpage
\setlength\parindent{24pt}
\begin{appendices}

\section{Dodatek}

Może się przyda

\end{appendices}

\end{document}